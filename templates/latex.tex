%==== THEOREM =====
%\documentclass[convert={density=300,size=1080x800,outext=.png}]{standalone}
%
%\usepackage{tcolorbox}
%\tcbuselibrary{theorems}
%\newtcbtheorem[number within=section]{mytheo}{Algebra}%
%{colback=green!5,colframe=green!35!black,fonttitle=\bfseries}{th}
%
%\begin{document}
%\begin{mytheo*}{Potenzregeln}{Anweisung: $$a^2+a$$}
%
%\end{mytheo*}
%\end{document}

\documentclass{standalone}
\usepackage{amsmath}
\usepackage{xcolor}
\usepackage[utf8]{inputenc} % to compile umlaute
\usepackage[breakable,skins]{tcolorbox}
\usepackage{xcolor}
\newtcolorbox{mybox}[1]{
before skip=1ex,
after skip=1ex,
top=2.5ex,
bottom=2.5ex,
width=18em,
breakable, 
enhanced,
coltitle=black,
colback=white,
sharp corners,
title={#1},
attach boxed title to top left={
    yshift=-\tcboxedtitleheight,
    xshift=\tcboxedtitleheight
    },
boxed title style={
    size=small,
    colback=white,
    sharp corners,
    }
}

\begin{document}
\begin{mybox}
{Brüche}\vspace{1.25em}Rechne aus: $$a^b * a^c$$
Ist das das selbe wie $a^{(b+c)}?$

%==== THEOREM =====
%\documentclass{standalone}
%\usepackage{amsmath}
%\usepackage[utf8]{inputenc} % to compile umlaute
%\usepackage[breakable,skins]{tcolorbox}
%\usepackage{xcolor}
%\newtcolorbox{mybox}[1]{
%before skip=1ex,
%after skip=1ex,
%top=2.5ex,
%bottom=2.5ex,
%width=18em,
%breakable, 
%enhanced,
%coltitle=black,
%colback=white,
%sharp corners,
%title={#1},
%attach boxed title to top left={
%    yshift=-\tcboxedtitleheight,
%    xshift=\tcboxedtitleheight
%    },
%boxed title style={
%    size=small,
%    colback=white,
%    sharp corners
%    }
%}
%
%\begin{document}
%\begin{mybox}
%    {Topic}
%\vspace{1.25em}
%Instruction:\\
%
%$$a^2=1$$
%Beispiel
%\textbf{Adding} a number to the \textbf{whole function} translates the graph in the \textbf{y-direction}.
%\begin{itemize}
% \item If $a > 0$, the graph goes \textbf{upwards}
% \item If $a < 0$, the graph goes \textbf{downwards}
% \item This can be described by a \textbf{column vector}: $\begin{pmatrix}a\\
% \end{matrix}$.
%\end{itemize}
\end{mybox}
\end{document}

%% grafik
%\documentclass{standalone}
%\usepackage{pgfplots}
%
%\begin{document}
%\begin{tikzpicture}
%  \begin{axis}[
%    xlabel={$x$},
%    ylabel={$y$},
%    axis lines=middle,
%    xmin=0, xmax=5,
%    ymin=0, ymax=5,
%    xtick={0,1,2,3,4,5},
%    ytick={0,1,2,3,4,5},
%    grid=both,
%    minor tick num=1,
%    ]
%    \addplot[blue,domain=-5:5] {2*x + 1};
%    \draw[red, thick] (axis cs: 0,1) -- (axis cs: 1,3) -- (axis cs: 1,1) -- cycle;
%    \node[below right] at (axis cs: 1.5, 4.5) {$y = mx + b$};
%    \node[right] at (axis cs: 1.5,2) {Steigung: $m=\frac{\Delta y}{\Delta x}$};
%    \node[above] at (axis cs: 0.5, 0.5) {$\Delta x$};
%    \node[left] at (axis cs: 1.5, 2) {$\Delta y$};
%  \end{axis}
%\end{tikzpicture}
%\end{document}
